% Appendix: Verified Statements Map
% Add this to your paper.tex

\appendix

\section{Formal Verification: Lean 4 Correspondence}
\label{app:lean}

All theorems in this paper have been formally verified in Lean 4 using the Mathlib library. The complete source code is available at:
\begin{center}
\url{https://github.com/paolovella/DefectCRN}
\end{center}

\noindent Version \texttt{v2.0.0} (commit \texttt{1063836}) contains 3685 lines of Lean 4 code with 98 theorems, zero \texttt{sorry} statements, and no custom axioms beyond Lean's core type theory.

\subsection{Main Theorem Correspondence}

Table~\ref{tab:lean-correspondence} maps the principal results of this paper to their formal Lean declarations.

\begin{table}[htbp]
\centering
\caption{Paper theorems and their Lean 4 formalizations.}
\label{tab:lean-correspondence}
\small
\begin{tabular}{@{}clll@{}}
\toprule
\textbf{\#} & \textbf{Paper Result} & \textbf{Lean Declaration} & \textbf{File} \\
\midrule
1 & Thm.~2.3 (Hodge Decomposition) & \texttt{hodge\_decomp} & \texttt{Basic.lean} \\
2 & Thm.~2.5 (Laplacian Inverse) & \texttt{laplacian\_inverse\_exists} & \texttt{Basic.lean} \\
3 & Thm.~4.2 (Onsager Optimality) & \texttt{onsager\_rayleigh\_optimal} & \texttt{Basic.lean} \\
4 & Cor.~4.3 (Onsager Uniqueness) & \texttt{onsager\_rayleigh\_unique} & \texttt{Basic.lean} \\
5 & Thm.~4.5 (Lyapunov Function) & \texttt{lyapunov\_zero\_iff} & \texttt{Basic.lean} \\
6 & Thm.~5.10 (Cycle Affinity) & \texttt{cycle\_affinity\_constant} & \texttt{CRNT.lean} \\
7 & Thm.~5.11 (Deficiency Zero) & \texttt{deficiency\_zero\_equilibrium\_exists} & \texttt{CRNT.lean} \\
8 & Thm.~5.12 (Deficiency One) & \texttt{deficiencyOne\_existence} & \texttt{DeficiencyOne.lean} \\
9 & Thm.~6.1 (Persistence) & \texttt{deficiency\_zero\_persistence} & \texttt{Persistence.lean} \\
10 & Thm.~6.5 (Product Form) & \texttt{product\_form\_is\_stationary} & \texttt{Stochastic.lean} \\
11 & Thm.~7.8 (Michaelis--Menten) & \texttt{michaelis\_menten\_velocity} & \texttt{MichaelisMenten.lean} \\
\bottomrule
\end{tabular}
\end{table}

\subsection{Module Structure}

The formalization is organized into the following modules:

\begin{table}[htbp]
\centering
\caption{Lean module structure and verification statistics.}
\label{tab:lean-modules}
\small
\begin{tabular}{@{}lrrl@{}}
\toprule
\textbf{Module} & \textbf{Lines} & \textbf{Theorems} & \textbf{Content} \\
\midrule
\texttt{Basic.lean} & 852 & 38 & Laplacian, Hodge, Onsager--Rayleigh \\
\texttt{CRNT.lean} & 512 & 10 & Stoichiometry, deficiency, mass-action \\
\texttt{DeficiencyOne.lean} & 367 & 4 & Deficiency one theorem \\
\texttt{Persistence.lean} & 312 & 8 & Persistence, permanence, $\omega$-limits \\
\texttt{Stochastic.lean} & 241 & 6 & Chemical Master Equation \\
\texttt{Examples/*.lean} & 1401 & 32 & Triangle, Cycle, Michaelis--Menten, Glycolysis \\
\midrule
\textbf{Total} & \textbf{3685} & \textbf{98} & \\
\bottomrule
\end{tabular}
\end{table}

\subsection{Verification Guarantees}

The Lean type checker provides the following guarantees:
\begin{itemize}
    \item \textbf{Soundness}: Every theorem statement is derivable from the axioms of Lean's type theory (CIC with inductive types).
    \item \textbf{Completeness of proofs}: No \texttt{sorry} (proof placeholder) appears in the codebase.
    \item \textbf{No additional axioms}: Beyond Lean's core (propositional extensionality, quotient types, choice), no custom axioms are introduced.
\end{itemize}

\subsection{Building and Verifying}

To independently verify all proofs:
\begin{verbatim}
git clone https://github.com/paolovella/DefectCRN.git
cd DefectCRN
git checkout v2.0.0
lake exe cache get   # Download Mathlib cache
lake build           # Verify all proofs (exit code 0 = success)
\end{verbatim}

\subsection{Citation}

If using this formalization, please cite:
\begin{verbatim}
@software{vella2025defectcrn,
  author  = {Vella, Paolo},
  title   = {{DefectCRN}: Formal Verification of
             Chemical Reaction Network Theory},
  year    = {2025},
  url     = {https://github.com/paolovella/DefectCRN},
  version = {2.0.0}
}
\end{verbatim}
